% Options for packages loaded elsewhere
\PassOptionsToPackage{unicode}{hyperref}
\PassOptionsToPackage{hyphens}{url}
%
\documentclass[
]{article}
\usepackage{amsmath,amssymb}
\usepackage{iftex}
\ifPDFTeX
  \usepackage[T1]{fontenc}
  \usepackage[utf8]{inputenc}
  \usepackage{textcomp} % provide euro and other symbols
\else % if luatex or xetex
  \usepackage{unicode-math} % this also loads fontspec
  \defaultfontfeatures{Scale=MatchLowercase}
  \defaultfontfeatures[\rmfamily]{Ligatures=TeX,Scale=1}
\fi
\usepackage{lmodern}
\ifPDFTeX\else
  % xetex/luatex font selection
\fi
% Use upquote if available, for straight quotes in verbatim environments
\IfFileExists{upquote.sty}{\usepackage{upquote}}{}
\IfFileExists{microtype.sty}{% use microtype if available
  \usepackage[]{microtype}
  \UseMicrotypeSet[protrusion]{basicmath} % disable protrusion for tt fonts
}{}
\makeatletter
\@ifundefined{KOMAClassName}{% if non-KOMA class
  \IfFileExists{parskip.sty}{%
    \usepackage{parskip}
  }{% else
    \setlength{\parindent}{0pt}
    \setlength{\parskip}{6pt plus 2pt minus 1pt}}
}{% if KOMA class
  \KOMAoptions{parskip=half}}
\makeatother
\usepackage{xcolor}
\usepackage{graphicx}
\makeatletter
\def\maxwidth{\ifdim\Gin@nat@width>\linewidth\linewidth\else\Gin@nat@width\fi}
\def\maxheight{\ifdim\Gin@nat@height>\textheight\textheight\else\Gin@nat@height\fi}
\makeatother
% Scale images if necessary, so that they will not overflow the page
% margins by default, and it is still possible to overwrite the defaults
% using explicit options in \includegraphics[width, height, ...]{}
\setkeys{Gin}{width=\maxwidth,height=\maxheight,keepaspectratio}
% Set default figure placement to htbp
\makeatletter
\def\fps@figure{htbp}
\makeatother
\setlength{\emergencystretch}{3em} % prevent overfull lines
\providecommand{\tightlist}{%
  \setlength{\itemsep}{0pt}\setlength{\parskip}{0pt}}
\setcounter{secnumdepth}{-\maxdimen} % remove section numbering
% definitions for citeproc citations
\NewDocumentCommand\citeproctext{}{}
\NewDocumentCommand\citeproc{mm}{%
  \begingroup\def\citeproctext{#2}\cite{#1}\endgroup}
\makeatletter
 % allow citations to break across lines
 \let\@cite@ofmt\@firstofone
 % avoid brackets around text for \cite:
 \def\@biblabel#1{}
 \def\@cite#1#2{{#1\if@tempswa , #2\fi}}
\makeatother
\newlength{\cslhangindent}
\setlength{\cslhangindent}{1.5em}
\newlength{\csllabelwidth}
\setlength{\csllabelwidth}{3em}
\newenvironment{CSLReferences}[2] % #1 hanging-indent, #2 entry-spacing
 {\begin{list}{}{%
  \setlength{\itemindent}{0pt}
  \setlength{\leftmargin}{0pt}
  \setlength{\parsep}{0pt}
  % turn on hanging indent if param 1 is 1
  \ifodd #1
   \setlength{\leftmargin}{\cslhangindent}
   \setlength{\itemindent}{-1\cslhangindent}
  \fi
  % set entry spacing
  \setlength{\itemsep}{#2\baselineskip}}}
 {\end{list}}
\usepackage{calc}
\newcommand{\CSLBlock}[1]{\hfill\break\parbox[t]{\linewidth}{\strut\ignorespaces#1\strut}}
\newcommand{\CSLLeftMargin}[1]{\parbox[t]{\csllabelwidth}{\strut#1\strut}}
\newcommand{\CSLRightInline}[1]{\parbox[t]{\linewidth - \csllabelwidth}{\strut#1\strut}}
\newcommand{\CSLIndent}[1]{\hspace{\cslhangindent}#1}
\usepackage{setspace}
\doublespacing
\usepackage[vmargin=1in,hmargin=1in]{geometry}
\usepackage{graphicx}
\usepackage{subfig}
\usepackage{tabularx}
\usepackage{float}
\usepackage{pdflscape}
\usepackage{enumitem} 
\usepackage[running]{lineno}
\usepackage{cite}
\linenumbers
\ifLuaTeX
  \usepackage{selnolig}  % disable illegal ligatures
\fi
\IfFileExists{bookmark.sty}{\usepackage{bookmark}}{\usepackage{hyperref}}
\IfFileExists{xurl.sty}{\usepackage{xurl}}{} % add URL line breaks if available
\urlstyle{same}
\hypersetup{
  hidelinks,
  pdfcreator={LaTeX via pandoc}}

\author{}
\date{}

\begin{document}

\section{Modelling Amazon fire regimes under climate change
scenarios}\label{modelling-amazon-fire-regimes-under-climate-change-scenarios}

Leonardo A. Saravia \textsuperscript{1} \textsuperscript{5}, Korinna T.
Allhoff \textsuperscript{2} \textsuperscript{3}, Ben Bond-Lamberty
\textsuperscript{4}, Samir Suweis \textsuperscript{5}

\begin{enumerate}
\def\labelenumi{\arabic{enumi}.}
\item
  Centro Austral de Investigaciones Científicas (CADIC-CONICET),
  Ushuaia, Argentina.
\item
  University of Hohenheim, Institute of Biology, FG Eco-Evolutionary
  Modelling (190m), Stuttgart, Germany
\item
  KomBioTa -- Center for Biodiversity and Integrative Taxonomy,
  University of Hohenheim \& State Museum of Natural History, Stuttgart,
  Germany
\item
  Pacific Northwest National Laboratory, Joint Global Change Research
  Institute, 5825 University Research Court \#3500, College Park, MD
  20740, USA
\item
  Laboratory of Interdisciplinary Physics, Department of Physics and
  Astronomy ``G. Galilei'', University of Padova, Padova, Italy
\item
  Corresponding author e-mail lasaravia@untdf.edu.ar, ORCID
  https://orcid.org/0000-0002-7911-4398
\end{enumerate}

\newpage

\subsection{Abstract}\label{abstract}

Fire is one of the most important disturbances of the earth-system,
shaping the biodiversity of ecosystems and particularly forests.
Climatic change and other anthropogenic drivers such as deforestation
and land use change could produce abrupt changes in fire regimes,
potentially triggering transition from forests to savannah or grasslands
ecosystems with large accompanying biodiversity losses. The interplay
between climate change and deforestation might intensify fire ignition
and spread, potentially giving rise to more extensive, intense, and
frequent fires, but this is highly uncertain. We use a simple
forest-fire model to analyze the possible changes in the Amazon region's
fire regime that depend on climate change-related variables. We first
explored the model behavior and found that there are two possible regime
changes: a critical regime that implies high variability in fire
extension and mega-fires, and an absorbing phase transition which would
produce the extinction of the forest and transition to a different
vegetation state. We parameterize the model using remote sensing data on
fire extension and temperature, and show that it demonstrates
proficiency in predicting past fires. Upon considering 21st-century
climate projections and deforestation scenarios, our findings suggest
that the Amazon region is not currently nearing any of these regime
changes but predict a consistent increase in fire extension mainly
induced by deforestation. Therefore, stopping deforestation could be an
important factor in reducing the potential for drastic alterations in
tropical forests of the Amazon region.

\subsection{Introduction}\label{introduction}

Few regions of the terrestrial biosphere are unaffected by fire. Fires
caused directly or indirectly by human activities (Bowman et al. 2020)
have different characteristics from natural fires, including in spatial
pattern, severity, burn frequency and seasonality, producing contrasting
ecological consequences (Steel et al. 2021). Recent years have seen an
increase in fire intensity and extension in different regions (Bowman et
al. 2020, Pivello et al. 2021), partially attributable to the fact we
are experiencing a biosphere temperature that is 1°C above historical
records (Masson-Delmotte et al. 2021); it is hypothesized that this
intensification could reduce the spatial and temporal variation in fire
regimes, called pyrodiversity (Kelly and Brotons 2017), which in turn
will generate substantial reductions in biodiversity and ecosystem
processes such as carbon storage (Dieleman et al. 2020, Furlaud et al.
2021). The regions most affected are likely the ones in which fire has
historically been rare or absent. In regions such as tropical forests
(Barlow et al. 2020), extreme fires could trigger extensive biodiversity
loss as well as major ecosystems changes such as transitions from forest
to savannah or shrublands (Hirota et al. 2011, Fairman et al. 2015).

Fires in the Amazon region were historically rare, due to the ability of
old-growth forest to maintain enough moisture to prevent fire spread,
even after prolonged drought periods (Uhl and Kauffman 1990). Human
activities, specifically deforestation and resulting land-use changes
over the past four decades, have created conditions conducive to more
frequent and widespread fires across the basin (Alencar et al. 2011,
Aragão et al. 2018, Cardil et al. 2020). Another significant factor
contributing to this phenomenon is the predicted increase in droughts
due to climatic change. Droughts can interact with deforestation,
potentially exacerbating land cover conversion and creating a dangerous
positive feedback loop (Barlow et al. 2020). Furthermore, human
activities such as secondary vegetation slash-and-burn and cyclical
fire-based pasture cleaning can serve as ignition sources for forest
fires (Aragão et al. 2018). Despite substantial reductions in
deforestation rates until 2018 (Feng et al. 2021), previous
deforestation activities may still provide sufficient ignition sources
for fires to expand into adjacent forests (Aragão et al. 2018). This
process could raise the importance of fires unrelated to deforestation
(Aragão et al. 2014).

Different models of fire for the Amazon have been developed to predict
regime changes under climate change scenarios. Here, we define a regime
change as an abrupt transition in fire patterns across large areas, such
as a shift from infrequent scattered fires to more frequent and
extensive burning (Kelly et al. 2020). These models can be process-based
(Le Page et al. 2017) or statistical (Fonseca et al. 2019), and
generally consider land-use change and other human activities, as well
as local weather conditions, but they usually neglect the spatial
dynamics of fire spread. Statistical fire models take into account
mainly environmental factors (Turco et al. 2018), while process-oriented
models include more mechanistic details (Thonicke et al. 2010), and a
few treat spatial dynamical phenomena (Schertzer et al. 2014). Such
spatial dynamics are important because they can provide insights into
how local interactions give rise to emergent fire patterns (Pueyo et al.
2010), and potentially change the stability characteristics of the
entire dynamical system (Levin and Durret 1996).

Simple models of fire have been used as an example of self-organized
criticality (SOC), where systems can self-organize into a state
characterized by power-laws in different model outputs. For example, the
forest fire model of Drossel and Schwabl (1992) (DSM) was proposed to
show SOC in relation to the size distribution of disturbance events
(Jensen 1998). Power-laws imply scale invariance, meaning that there is
no characteristic scale in the model. Later it was shown that DSM does
not exhibit true scale invariance (Grassberger 2002) and that the system
needs to be somewhat tuned to observe criticality (Bonachela and Muñoz
2009). These facts decreased its theoretical attractiveness, but the
model could still be of high practical relevance. Some modifications of
the Drossell \& Schwabl (DSM) model have been used to predict fire
responses to climate change (Pueyo 2007), and other DSM variants can
reproduce features observed in empirical studies (Ratz 1995) such as the
power-law distributions of the fire sizes, the size and shape of
unburned areas and the relationship between annual burned area and
diversity of ecological stages (Zinck and Grimm 2009). An analysis of
different models showed that the key for reproducing all these patterns
was changing the scale of grid cells to represent several hectares, and
the `memory effect': flammability increases with the time since the last
fire at a given site (Zinck and Grimm 2009, Zinck et al. 2011). However,
the exponent of the fire size distribution observed in different
ecoregions cannot still be reproduced by these models.

The simple models mentioned above could have critical behaviour
characterized by a power-law distribution in fire sizes and other model
outputs. Such dynamics can be explained in terms of percolation theory
(Stauffer and Aharony 1994) where there is a transition between two
states: one where propagation of fires occurs, and another where it is
very limited. The narrow region where the transition occurs is the
critical point, characterized by an order parameter (fire size) that
depends on some external control parameter (e.g.~ignition probability)
(Solé and Bascompte 2006). An example of this transition could be the
case of the recent Australia 2019-2020 mega-fires (Nolan et al. 2020,
Nicoletti et al. 2023), which had devastating consequences for
biodiversity and ecosystem functioning (Kelly et al. 2020).
Historically, indigenous fire stewardship in Australian landscapes
maintained flammable forest in a disconnected state by producing
frequent small scale fires ({``Biodiversity in flames''} 2020), thus
preventing high-intensity fires and protecting biodiversity. This regime
was disrupted by fire suppression related to European colonization
land-use change (Hoffman et al. 2021) and climate (Adams et al. 2020),
pushing the system toward a critical regime (Nicoletti et al. 2023) with
less ecosystem resilience to extreme fires. These extreme events are
very difficult to predict by Earth system models that do not fully
incorporate the dynamic of fuel accumulation and vegetation dynamics or
their effects on biodiversity and ecosystem services (Sanderson and
Fisher 2020).

The objective of this work is to model and predict the changes in fire
regimes in the Amazon region using a simple spatial stochastic fire
model, based on variables like precipitation and temperature that are
included in climate change scenarios. We thus assume that the emergent
dynamics can be described by a process of slow accumulation of fuel and
rapid discharge produced by the fires. More precisely, we initially
reconstruct past fire patterns using the NASA Moderate-Resolution
Imaging Spectroradiometer (MODIS) burnt area product. This step serves
to verify our main assumption and to derive an ignition probability.
Subsequently, in the second step, we predict the ignition probability up
to the year 2060 based on different greenhouse gas Representative
Concentration Pathways. The third step involves simulations to explore
the behavior of the Fire model. In the fourth step, we fit the Fire
Model to the observed MODIS fire patterns. Finally, utilizing the model
forced with the ignition probability, we predict and analyze potential
changes in the modeled fire regimes for the Amazon, taking into account
the influence of deforestation on model parameters.

\subsection{Methods}\label{methods}

Our study region is the Amazon biome (Figure S1). This includes Brazil,
which represents 60\% of the area, as well as eight other countries
(Bolivia, Colombia, Ecuador, Guyana, Peru, Suriname, Venezuela, and
French Guiana). We chose this region because a significant amount of
fires extend to tropical moist forests outside Brazil (Cardil et al.
2020), and the whole area is thought to be a crucial tipping element of
the Earth-system (Staver et al. 2011, Lenton and Williams 2013).

\subsubsection{Reconstruction of past fire patterns from MODIS
data}\label{reconstruction-of-past-fire-patterns-from-modis-data}

We estimated the monthly burned areas from 2001 to the end of 2021 using
the NASA Moderate-Resolution Imaging Spectroradiometer (MODIS) burnt
area Collection 6 product MCD64A1 (Giglio et al. 2016), which has a 460
m pixel resolution. We used Google Earth Engine with the JavaScript
programming language to download the data restricted to the region of
interest (see source code availability below). Each image represents the
burned pixels as 1 and the non-burned as 0. We then calculated the
burned clusters using four nearest neighbours (Von Neumann
neighbourhood) and the Hoshen--Kopelman algorithm (Hoshen and Kopelman
1976). Each cluster contains contiguous pixels burned within a month and
this represents a fire event \(S\), allowing us to calculate the number
and sizes of fire clusters by month. We estimated the probability of
ignition \(f\) as \(f(t) = \frac{|S_t|}{T}\), where \(|S_t|\) denotes
the number of clusters \(S_t\) that start in the month \(t\) (if a fire
started in the previous month we avoided it to remove possible double
counting), and \(T\) is total number of pixels in the region, to allow
comparisons with the fire model.

We also estimated the distribution of fire sizes using an annual period
to have enough fire clusters to discriminate between different
distributions. We aggregated the monthly images using a simple
superposition; the annual image has a 1 if it has one or more fires
during the year, and 0 if it has none. This assumes that most of the
sites burn only once a year, we verified this using the MODIS data that
on average only 0.06\% burn more than once annually. After that, we ran
again the Hoshen--Kopelman algorithm to obtain the annual fire clusters.
Then we fitted the following distributions to the fire sizes: power-law,
power-law with exponential cut-off, log-normal, and exponential. We used
maximum likelihood to decide which distribution fit the data best using
the Akaike Information Criteria (\(AIC\)) (Clauset et al. 2009).
Additionally, we computed a likelihood ratio test, Voung's test (Vuong
1989), for non-nested models. We only considered it a true power-law
when the value of the \(AIC\) was at a minimum and the comparison with
the exponential distribution using the Vuong's test was significant with
p\textless0.05; if p\textgreater=0.05 we assumed that the two
distributions cannot be differentiated.

\subsubsection{Fitting of ignition
parameter}\label{fitting-of-ignition-parameter}

We calculated the monthly ignition probability \(f\) and related it to
monthly precipitation (\(ppt\)), maximum temperature (\(T_{max}\)) and a
seasonal term (\(m\)). These variables have generally been included in
global and regional fire activity models (Huang et al. 2015, Turco et
al. 2018, Fonseca et al. 2019, Wei et al. 2020). More variables were
used in these models, but we are constrained by the variables available
in the Climate Projections (see below). We obtained environmental data
from the TerraClimate dataset (Abatzoglou et al. 2018), averaging over
the study region, to represent the influence of regional climate over
\(f\). We evaluated an increasingly complex series of generalized
additive models (GAMs), assuming a Gaussian distribution and transformed
\(f\) to logarithms, because it had a highly skewed distribution. We
also fitted the same models assuming a Gamma distribution and no
transformation for \(f\). For all the models we used thin plate
regression splines (Pedersen et al. 2019) as smoothing terms, and for
interactions between environmental variables we used tensor products,
using restricted maximum likelihood (REML) to fit to the data (Pedersen
et al. 2019). All these procedures were available in the R package
\textbf{mgcv} (Wood 2017) and all source code is available in the
repository \url{https://github.com/lsaravia/AmazonFireTippingPoints}.

We selected the best models using \(AIC\) (Wood 2017). To evaluate the
predictive power of the models, we split the data set into a training
set representing 85\% of the data and testing set (always 3 years long)
and repeated the procedure 10 times starting at different random dates.
We then calculated the mean absolute error (MAE) and the Root Mean
Squared Error (RMSE) for the three best models selected with \(AIC\)
(Table S2). The formulas of MAE and RMSE are as follows:

MAE = \(\frac{1}{n} \sum_{i=1}^n |f_i - \hat{f_i}|\)

RMSE = \(\sqrt{\frac{1}{n} \sum_{i=1}^n (f_i - \hat{f_i})^2}\)

Where \(f_i\) is the observed ignition probability \(f\) at month \(i\),
\(\hat{f_i}\) the predicted \(f\) and \(n\) the total number of months
used for predictions. We selected the model with the smallest Mean
Absolute Error (MAE) and Root Mean Square Error (RMSE) to generate
predictions of the ignition probability up to 2060. These predictions
serve as input for the fire model, as explained in more detail below.
Driving data were obtained from the NASA Earth Exchange Global Daily
Downscaled Climate Projections NEX-GDPP
(https://www.nccs.nasa.gov/services/data-collections/land-based-products/nex-gddp)(Thrasher
et al. 2012), which were estimated with General Circulation Models (GCM)
runs conducted under the Coupled Model Intercomparison Project Phase 5
(Taylor et al. 2012). We averaged over the 21 CMIP5 models and over the
study region to obtain the monthly values of the variables needed:
precipitation and maximum temperature. Then we estimated the probability
of ignition up to 2060 using the fitted GAM across two of the four
Representative Concentration Pathways (RCPs), RCP4.5 and RCP8.5
(Meinshausen et al. 2011). Such RCPs are greenhouse gas concentration
trajectories adopted by the IPCC and used for climate modelling and
research (Moss et al. 2010). RCP4.5 represents an intermediate scenario,
indicating a peak in emissions around 2040 followed by a decline. On the
other hand, RCP8.5 depicts a scenario where emissions persistently
increase throughout the 21st century.

\subsubsection{Fire Model definition and
exploration}\label{fire-model-definition-and-exploration}

Conceptually the model represents two key biological processes: forest
burning and forest recovery. We assume that the forest layer represents
the flammable biomass or fuel layer of the forest (rather than absolute
forest cover), as the focus is on modeling fire spread dynamics. When a
site is burned in the model, it does not necessarily mean that all
vegetation is killed - rather, it represents the consumption of
flammable fuels. The regrowth or recovery process simulates vegetation
re-establishing flammable biomass over time.

The model uses a two-dimensional lattice to represent the spatial
region. Each site in the lattice can be in one of three different
states: an empty or burned site, a flammable forest (called forest for
short), or a burning forest. The lattice is updated in parallel,
according to the following steps:

\begin{enumerate}
\def\labelenumi{\arabic{enumi}.}
\item
  We pick at random a burning site, and it becomes an empty site in the
  following step (the model's timestep is one day)
\item
  We pick at random a forest site and it becomes a burning forest if one
  or more of its four nearest neighbour sites are burning
\item
  We pick at random another forest site and it sends (with probability
  \(p\)) a propagule to an empty site at a distance drawn from a
  power-law dispersal kernel with exponent \(de\).
\item
  A random site can catch fire spontaneously with probability \(f\),
  i.e., which varies monthly to reflect the fire season. This parameter
  is fitted to fire data for predictions, global change scenarios, and
  deforestation scenarios (refer to the previous section for details).
\end{enumerate}

We assumed absorbing boundary conditions and a lattice size of 450x450
sites, but also ran simulations with other sizes, resulting in
equivalent results. Rule 3 means that a burned or empty site can become
forest more quickly when it is near a forest site, but also that some
sites can become forest even when far from established forest
sites--depending on the kernel exponent, it could be any site in the
lattice (Marco et al. 2011). The choice of a power-law dispersal is
justified because forests dispersion generally exhibits fat-tailed
kernels (Clark et al. 2005, Seri et al. 2012).

This model is very similar to the Drossel-Schwabl (DSM) forest fire
model (Drossel and Schwabl 1992): it exhibits critical behaviour when
\(\theta = p/f\) tends to \(\infty\), and thus must satisfy the
condition that \(f << p\). This condition means that the chances of an
empty site turning into forest are much higher than the chances of a
healthy forest site catching fire spontaneously, as is generally
observed in natural systems. The model involves the separation between
three time scales: the fast burning of forest clusters, the slow
recovery of forest, and the even slower rate of fire ignition. Then in
the critical regime there is a slow accumulation of forest that forms
connected clusters, and eventually as the ignition probability is very
low these clusters connect the whole lattice--- here is the link with
percolation theory (Stauffer and Aharony 1994)--- and a single ignition
event can produce large fires. After this, the density of the forest
becomes very low and the accumulation cycle begins again. This regime is
characterized by wide fluctuations in the size of fires and the density
of trees, with both following approximately power-law size
distributions. If the ignition probability \(f\) is too high fires are
frequent, forest sites become disconnected and small fires, of
characteristic size, dominate the system.

One of the features not present in the original forest fire model is
that forests can have long-distance dispersal; ecologically, this
determines population spread, the colonization of empty habitats, and
the assembly of local communities (Nathan et al. 2008). The inclusion of
long distance dispersal can modify the distribution of forest clusters,
the distribution of fire sizes, and the dynamics of the model (Marco et
al. 2011). When forest dispersal is limited mainly to nearest
neighbours, forest recovery produces clusters that tend to coalesce and
form uniform clusters with few or no isolated forest sites. When the
forest burns, these isolated forest sites are the points from where the
forest recovers (assuming no external colonization); when these are not
present there is an increased probability that the forest becomes
extinct. With long-distance dispersal, there is an important number of
isolated forest sites, thus decreasing the probability of forest
extinction (Figure S13). These processes are particularly important when
\(\theta\) is low and fires are smaller but more frequent. In dynamical
terms there is a critical extinction value \(\theta_{ext}\), when
\(\theta < \theta_{ext}\) the forest become extinct. However, the
critical value depends on the dispersal distance governed by \(d_e\). At
higher \(d_e\), the dispersal distance is smaller, the forest patches
are more compact and isolated, leading to a higher probability of
complete burning and a lower \(\theta_{ext}\).

The second feature not present in the original forest fire model is
seasonality. In natural systems, certain times of the year exhibit
environmental conditions that significantly increase the probability of
fire ignition, while during the remaining periods, the probability of
fires is much lower. This results in a periodic cycle of forest
accumulation and a brief duration of intense fires, commonly known as
the fire season. Consequently, the model incorporates a short period
characterized by a low \(\theta_{min}\) and a longer period with a high
\(\theta_{max}\). When both the minimum and maximum \(\theta\) values
fall within the critical region, the model's long-term behavior
resembles the critical regime, with maximum fire sizes occurring during
the fire season. If \(\theta_{max}\) is within the critical region and
\(\theta_{min}\) is outside it, the model's dynamics may exhibit more
extreme fires (similar to the critical regime) compared to an equivalent
non-seasonal model. Conversely, when both \(\theta\) values are outside
the critical region, the dynamics may approach the critical extinction
zone. In this scenario, seasonal differences in fire sizes are less
pronounced.

Increasing the length of the fire season as predicted in climate change
scenarios (Pausas and Keeley 2021) will produce the model to spend more
time at a lower \(\theta\) decreasing the connectivity of the forest and
the size of fires. Moreover, this could increase the possibility of
critical extinction if both \(\theta_{max} and \theta_{min}\) are below
\(\theta_{ext}\). In this work we assume that the forest is flammable
forest; the extinction of this state could mean that environmental
conditions become wetter and the forest does not burn anymore. If
environmental conditions become drier, the extinction of forest probably
means a transition to another type of vegetation and then the conditions
to apply this model will no longer hold.

We conducted a set of exploratory simulations, with a range of
parameters compatible with what we found for the Amazon region, to
characterize the regimes described above (Table S3). Using a lattice
size of 450x450 sites, we ran the simulations for 60 years with an
initial forest density of 0.3 (we found that different initial
conditions gave similar results) and used the final 40 years to estimate
the total annual fire size, the maximum cluster fire size, the
distribution of fire sizes and the total number of fires. To determine
the cluster fire sizes and distributions we used the same methods
described previously for the MODIS fire data. We ran a factorial
combination of dispersal exponent \(de\) and \(\theta\) and 10
repetitions of each parameter set. First, we ran the experiment with
\(\theta\) fixed, keeping the ignition probability \(f\) constant, and
then repeated the experiment with seasonality: we simulated a fire
season of 3 months each year multiplying \(f\) by 10. A dispersal
exponent \(de>>1\) (e.g.~\(de=102\)) is equivalent to a dispersal to the
nearest neighbours, while \(de=2.0155\) corresponds to a mean dispersal
distance of 66 sites (Table S3), i.e.~long range dispersal.

\subsubsection{Fire Model Fitting}\label{fire-model-fitting}

As we already estimated the \(f\) parameter from the 21 years of MODIS
data, we only needed to estimate the dispersal exponent \(de\) and the
probability \(p\) of forest growth. This parameter \(p\) is expressed as
\(r=1/p\), representing the average number of days for forest to
recover. For this estimation we duplicated the extension of the
estimated \(f\) as if it started in 1980; we allowed 20 years for
transient effects to dissipate in the model, and then used the last 20
years to compare with monthly fire data. This choice was justified
because most human activities in the Amazon started in this decade
during the conversion of large areas of forest to agriculture (Brando et
al. 2019).

To explore the parameter space we used Latin-hypercube sampling (Fang et
al. 2005) with parameter ranges \(2.5 - 2.00035\) for \(de\) (equivalent
to a mean dispersal distance range of 3 - 290 sites) and \(365 - 7300\)
days for \(r\). As the model has a long transient period, we could
suppose that the system is in a transient state, and thus we also
estimated the initial forest density as a parameter with a range of
densities of 0.2 - 0.7. We used 600 samples and 10 repeated simulations
of the model for each sample, totalling 6000 simulations, and selected
the best parameter set using Approximate Bayesian Computation (ABC)
(Csilléry et al. 2010, 2012) by comparing the relative monthly fire size
--- i.e., the absolute size divided by the total number of sites (the
number of MODIS pixels for the Amazon basin or the number of lattice
sites for simulations) -- with model predictions using Euclidean
distance. During an initial test, we observed that the peaks in the
model were delayed by 2-3 months; the same happens in more realistic
process-based models (Thonicke et al. 2010), and as we were not
interested in predicting the exact seasonal fire patterns, so instead of
using the complete monthly time series we used the monthly maximum fire
size of the year. We validated this choice using a random model
simulation with known parameters as data and verifying that we could
recover the parameters (see source code). The second step of our fitting
procedure was to use power-law fire distribution to perform another ABC;
we thus ran 10 simulations for each of the parameter sets in the
posterior distribution and then calculated the fire cluster
distributions using the same methods explained previously. We used ABC
to select a second posterior parameter distribution compared to the
median value of the power-law exponent from MODIS data. The tolerance
(proportion of results accepted nearest the target values) for both ABC
procedures was set at 0.05.

Finally, we ran the model with the final posterior parameter
distribution, the ignition probability estimated from the MODIS data,
and the ignition probability estimated with the GAM model for the period
2000-2021, to check if the data fit the prediction range. We performed
the ABC using a lattice size of 450x450 sites, and as we used relative
values (e.g.~absolute fire size divided by the total number of sites)
the model does not represent a defined scale.

\subsubsection{Model Predictions}\label{model-predictions}

We used the set of posterior parameter distribution set and the
predictions of the parameter \(f\) under RCP4.5 and RCP8.5 to perform
simulations up to 2060. We started simulations in the year 1980 as in
the fitting procedure, but instead of using \(f\) derived directly from
data, we used \(f\) obtained from the GAM model, allowing us to compare
actual and predicted fires using the same method to obtain \(f\).

Our model also does not explicitly account for deforestation's impact;
however, we approximate this by dynamically adjusting parameters as
deforestation intensifies. Initially fitted to the entire region based
on a singular flammable forest type assumption, these parameters reflect
an average between two vegetation categories: 1) post-deforestation
vegetation, featuring higher ignition probability (\(f\)) and faster
recovery (associated with higher growth probability, \(p\)); and 2) the
original forest, with lower \(f\) and \(p\) values. Consequently,
escalating deforestation leads to an estimated increase in the average
\(p\) and \(f\) values beyond the fitted period (see supplementary
material for more details). Leveraging the Hansen remote sensing product
(Hansen et al. 2013) updated until 2021, we calculated the mean
deforestation rate for the Amazon biome. Subsequently, under RCP4.5 and
RCP8.5 scenarios, we demonstrate the potential impact of deforestation
on the model's predictions. The estimated increase in \(f\) post-2021 is
shown in Figures S19 \& S20.

\subsection{Results}\label{results}

\subsubsection{Fire patterns from MODIS
data}\label{fire-patterns-from-modis-data}

The monthly fires follow a strong seasonal pattern with a maximum
between September and October (Figure S2). We characterize the annual
fire regime using the total fire size (total burned area) and the
maximum fire cluster (the biggest fire event \(S_{max}\)). We note that
the years with highest \(S_{max}\) are also years with a large total
area size (Figure 1). The years 2007 and 2010 had the two highest
\(S_{max}\) and they also have a power-law distribution (Table S1,
Figures S3-S5). Power-law distributions are defined as \(c S^{-\alpha}\)
where \(c\) is constant, \(\alpha\) the exponent and have an extra
parameters: \(S=x_{min}\), which is the minimum value for which the
power-law holds. The constant \(c\) is given by the normalization
requirement (Newman 2005). In the dataset, only 6 out of 20 years
demonstrate fire sizes that conform to a power-law distribution (Table
S1, Figures S3-S5). It is noteworthy that some of these distributions
display a range {[}\(S_{max} - x_{min}\){]} with notably higher values
compared to years without power-law behavior. However, variability
exists, as certain years with power-law patterns exhibit a relatively
small range. This dual extremity mirrors a discernible pattern observed
in the fire model under investigation.

\begin{figure}
\centering
\includegraphics{figure/Amazon_Annual_TotSizeVsMaxSize_arrows.jpg}
\caption{Annual total fire size vs maximum fire size relative to Amazon
basin, estimated with MODIS burned area product. These observed data
exhibit cycles of loading and discharge, years with high fire extension
and big fire events---the upper right region of the figure---which are
followed by years of low fire extension and no extreme events in the
lower left region. A typical trajectory could be the years 2009, 2010
and 2011 where this cycle can be clearly observed.}
\end{figure}

\subsubsection{Fitting of ignition
parameter}\label{fitting-of-ignition-parameter-1}

We fitted GAM models for the ignition probability \(f\) with single
variables, combinations of two interacting variables, and lags of one
month, the best model with lower \(AIC\) and lower MAE and RMSE was the
Gaussian with the interaction \(T_{max} * m\) (Table S2, Figure S6). For
the GAM fitted to the complete dataset we observe that the model does
not capture the most extreme years of \(f\) (Figure S7), but the model
fitted for the first years (\textless{} 2018) predicted the rest of the
data well (Figure S8).

With the best-fitted GAM and the \(T_{max}\) from the NASA Earth
Exchange Global Daily Downscaled Climate Projections, we predicted the
monthly \(f\) starting from 2021 for two greenhouse gas emissions
scenarios: RCP4.5 and RCP8.5. For the fire model simulations we added
the GAM's predictions using the actual data previous to 2021, we can
observe that the temporal pattern of \(f\) before 2021 are seasonal but
more irregular and variable than the patterns after 2021 (Figures S9 \&
S10).

\subsubsection{Fire model exploration}\label{fire-model-exploration}

We ran the model for a range of the \(\theta = p/f\) parameter,
anticipating that larger values would produce critical behaviour,
consisting of large variability of fires between years and extremely
large cluster fire sizes that follow a power-law distribution. As
expected, we obtained a larger proportion of power-law distributions for
the biggest size of \(\theta\) (Table S4 \& S5), and particularly high
variability and extremely large fires (Figure 2). For simulations with
seasonality, we observed the expected decrease in the number of years
where the size of the fire clusters follows a power-law distribution,
also less variability and fewer extreme fires, because in these cases
\(\theta\) decreases for the fire season. Seasonality also had the
unexpected effect of increasing the frequency of power-law distribution
for \(\theta = 25\) with a bigger exponent than the ones for large
\(\theta\) (Table S5); this pattern was also observed in the MODIS data.

In the simulations with \(\theta = 25\) and 250 and with shorter
dispersal distances, the forest density tends to decrease and eventually
reaches zero, marking the absorbing phase transition reported for this
type of model (Nicoletti et al. 2023); this means that in these cases
the parameter \(\theta\) was below the critical point \(\theta_{ext}\)
and so produces forest extinction (Figure S11). Increasing the dispersal
distance produces higher forest density, while seasonality has the
opposite effect. In the case of high dispersal and low \(\theta\) and
seasonality, we are again below \(\theta_{ext}\) (Figures S12 \& S13).
Note once again that forest density is the so-called active component of
the model and represents the flammable forest.

\begin{figure}
\centering
\includegraphics{figure/FireNL_TotSizeVsMaxSize_dispersal_theta_season.png}
\caption{Total annual fire size vs.~max fire cluster for the Fire model.
\textbf{A} \& \textbf{B} Are simulations with fixed \(\theta\), where
\(\theta = p/f\), \(p\) the forest recover probability and \(f\) the
ignition probability. \textbf{A} Are simulations with dispersal exponent
\(de=102\), mean dispersal distance of 1 (equivalent to nearest
neighbours ) and \textbf{B} with \(de=2.0155\), mean dispersal distance
of 66 sites. \textbf{C} \& \textbf{D} Are simulations with a fire season
of 90 days where \(\theta\) is divided by 10 (the probability of
ignition \(f\) is multiplied by 10), and the same \(de\) as previously.}
\end{figure}

\subsubsection{Fire Model Fitting}\label{fire-model-fitting-1}

After the first ABC we obtained the first posterior distribution of
parameters (Table S6). The model generated simulations that closely
resemble the monthly MODIS fire estimates, despite being fit only using
annual maxima (Figure S14). Repeated simulations with the same set of
parameters revealed significant variation due to the stochastic nature
of the model dynamics (Figure S14 \& S15). A noticeable lag in the
model's monthly maxima compared to MODIS data (Figures S14 \& S15) may
be due to the lack of a fire spread velocity parameter in the model.
Despite this, when evaluating the total annual fire size, the model
produced intervals that encompassed the MODIS estimates (Figure S16).
Thus, the lag in the monthly maxima does not significantly impact our
goal of predicting the total annual fire.

For the second ABC we used the posterior obtained in the first step, and
calculated the power-law exponent of the model simulations; this new set
of parameters had a similar range as the first (Table S7), with
simulated power-law exponents near the observed value (Figure S17). The
average \(\theta\) of the final posterior distribution has a mean of 61
a range between 9 and 910, which is a low-intermediate range,
considering the parameter range we used for the model exploration. The
active state of the model, representing the amount of forest,
consistently displayed a median value of 0.2 across all scenarios
(Figure S18).

When examining predictions without temporal structure (Figure 3), both
models, one simulated with the ignition probability (\(f\)) calculated
from the data and the other with \(f\) estimated using the GAM model,
produced results consistent with the observed data range. Comparing
medians, the predicted total fire size closely matched the data (Figure
3A), while the predicted number of fires was slightly higher (Figure
3C). The maximum fire size (\(S_{max}\)) was moderately elevated (Figure
3B), and the power-law exponent (\(\alpha\)) of the fire size
distribution was lower (Figure 3D). It is worth noting that these
results are interconnected, as a lower \(\alpha\) signifies larger fire
events. The model exhibited an extended range with a few very large
fires, an outcome expected due to the nonlinear nature of the model and
the number of simulations (around 100 for each observed data).

\begin{figure}
\centering
\includegraphics{figure/Amazon_ModelVsData_SimulGAM.png}
\caption{Predictions of the fire model compared with observed data for
the years 2001-2021. We used best-fit parameters, the ignition
probability from MODIS, and the ignition probability from the estimated
GAM models (Simul GAM), and run 100 simulations of the model. To make
them comparable we divided all the outputs (except power-law exponent)
by the total number of pixels in the region/model; black points are the
medians.}
\end{figure}

\subsubsection{Fire model predictions}\label{fire-model-predictions}

The modelled temporal series, based on the \(f\) from the GAM model for
the 2001-2021 period, exhibits a 95th percentile interval that
encompasses almost all the fire data (with the exception of the 2010
extreme fire). The median of the series suggests a decreasing trend in
annual fires over this time period (Figure 4). This decrease, along with
inter-annual fluctuations, is also apparent in the observational fire
data shown in Figure 1. Specifically, the most extensive fires occurred
in the first decade from 2001-2010, while the subsequent period from
2011-2021 experienced substantially less extensive burning. After 2021,
the model utilizes General Circulation Model (GCM) predictions to
estimate the fire ignition \(f\), and a mean rate of deforestation from
data.

Under both RCP scenarios in the absence of deforestation, a consistent
pattern emerges with a gradual increase in the total annual fires.
However, the annual peak values observed in this scenario remain below
those recorded during the 2001-2021 period. In contrast, when
considering deforestation, a notable surge is evident in both the median
and extreme annual fire occurrences (Figure 4). While both RCP scenarios
exhibit comparable trends, there is an expansion in the range of annual
fires within the RCP8.5 scenario. This expansion is primarily driven by
lower minimum fire occurrences.

\begin{figure}
\centering
\includegraphics{figure/Amazon_TotSize_year_theta61_RCP45-85Def.png}
\caption{Time series of predictions of the fire model without (left) and
with (right) deforestation, compared with observed data for the years
2001-2060. We used best-fit parameters distribution, the ignition
probability derived from GAM models estimated using actual data up to
2021, and projections based on General Circulation Models (GCMs)
following two greenhouse gas emissions scenarios, namely Representative
Concentration Pathways (RCPs) 4.5 and 8.5 for subsequent years. The data
points represent actual observations, while the line denotes the median
of simulations with accompanying 95\% confidence interval bands. All
outputs are presented relative to the total area.}
\end{figure}

The trends depicted in the total annual versus maximum fire plot by
decade (Figure 5) align with the observations described above. The plot
encompasses all model simulations, highlighting the model's proficiency
in simulating extreme fires. However, an anticipated trend solely
attributable to climate change scenarios would indicate a decline in
extreme fires and total annual fire occurrences. Incorporating
deforestation into the analysis revealed a significant escalation in
both the maximum fire size and the overall total fire size. Figure 5
illustrates the trend observed for RCP 4.5, which is also similar to
that depicted in Figure S21 for RCP 8.5. Both RCP scenarios exhibit a
similar pattern.

\begin{figure}
\centering
\includegraphics{figure/Amazon_TotSizeVsMax_year_theta61_RCP45_Def.png}
\caption{Total annual size of fires vs maximum monthly fire size \%
relative to the area of the region, data, predictions and predictions
including deforestation. The data column was estimated using the MODIS
burned area product. The predictions by decade were estimated with a
fitted model using a monthly ignition probability calculated with data
from General Circulation Models under two greenhouse gas emissions
scenarios known as Representative Concentration Pathways (RCPs), here
only RCP4.5 and RCP8.5 is shown in the figure S21. For the years
2001-2021 the ignition probability was estimated from actual data.}
\end{figure}

\subsection{Discussion}\label{discussion}

Based on spatial forest-fire dynamics, the model fitted to actual data
successfully reproduces past fire patterns and provides insights into
the future dynamics of Amazon fires. The predictions up to year 2060
suggest that the Amazon fire dynamics are driven to an increase in total
annual and extreme fires mainly caused by deforestation. Surprisingly,
this increase was mainly driven by deforestation and did not occur under
a scenario of climatic forcing alone. In fact, in the latter case, the
model actually predicts a reduction of extreme annual fires.

This last result might be due to a double average that we used for the
General Circulation Models (GCM): we averaged the tempertature over the
spatial extension of the region and over all the 21 GCMs. When we used
the actual data ---also averaged over the region--- the model predicts
in the range of observed extreme fires. For this reason, we hypothesize
that increased temporal variability of the ignition probability could
produce an increase in the extreme fires not captured by the model due
to the smoothed nature of the CGM data. The increase in the median
tendency of the annual fires under deforestation scenarios seems a more
robust prediction, however. It is in line with the predicted increase in
fire weather conditions for the area (Abatzoglou et al. 2019).

The observed decrease in median total fire size during the period from
2002 to 2021, when the model was fitted to data, could be attributed to
the reduction in deforestation levels in the Brazilian Amazon, which
were 44\% lower in 2020 compared to the levels recorded from 1996 to
2005 (Silva Junior et al. 2020). One interesting application of this
model is its ability to provide insights into trends that extend beyond
the fluctuations of actual data.

A critical regime would imply far more extreme fires and an absorbing
phase transition that could signal an imminent forest-savanna
transition, without extreme fires but with more frequent fires. The
actual and predicted fire regime seems to lie between these regimes.
This model does not explicitly include deforestation or slash-and-burn
and other agricultural areas---these are implicitly represented in the
flammable forest state---and thus the continuous increase of these land
uses represented in the deforestation scenarios, combined with the
increase in the probability of ignition \(f\) and produce important
changes in the Amazonian fire regime.

Similar models have been used to fit fire data and determine if a given
system is in a critical regime. For example, Zinck et al. (Zinck et al.
2011) found that some regions of Canada have experienced a change in the
fire regime from a non-critical to critical. They argued that the
original Drossel-Schwabl model (DSM) did not give the correct values of
the power-law exponent of fire distributions, and thus modified the
model to represent fire propagation as a stochastic birth-death process.
This means modelling fire as a contact process (Oborny et al. 2007) that
develops over the forest sites; the same concept was further explored
concerning the recent Australian mega-fires (Nicoletti et al. 2023).
Here we took a different approach: as in the DSM our model is based on
deterministic spread of fire, and we added what we think are the minimal
processes needed for more realism: seasonality and forest dispersal
distance. We agree with Zink et al. (2011) that an extension of the
original DSM was needed to represent fire processes observed in
ecosystems, but we also argue that not all complexity can or should be
added. In fact, it is necessary to keep the model tractable in order to
e.g.~perform parameter-space explorations. Our model is consequently
phenomenological, in the sense that it does not include all mechanisms
present at local scales but still tries to predict fire dynamics at
broad scales. Our results identify deforestation as a key driving force
for changes in the fire regime. We therefore suggest that future model
versions including a deforested state with different ignition \(f\) and
recovery \(p\) probabilities should be built to allow for a rigorous
comparision between these types of fire models. Such a comparison
clearly goes beyond the scope of the present study but would be needed
to gain deeper insights into the role of deforestation.

One of the advantages of this kind of model is that it can be applied to
different systems. This is the case for the original DSM model which has
been applied to brain activity and rainfalls (Palmieri and Jensen 2020).
In these two systems there are cycles of loading and discharge, a broad
region where the fluctuations peak as the critical behaviour is
established, and not a critical point with a very sharp transition as
the theory of second-order phase transitions suggests (Stauffer and
Aharony 1994). A similar behaviour occurs in our model version, but not
in the original DSM, due to the temporal dependence of the parameters
imposed by the fire seasons, where during some months there is a higher
ignition probability \(f\). This changes the control parameters with
respect to the DMS model since, in fact, we do not observe a transition
for a specific value of the \(\theta\) parameter. Our results suggest
that instead of having a specific fine-tuning to observe critical fire
spread, a critical region similar to a ``Griffiths phase'' may be
present in our model. Originally defined in statistical physics (Moretti
and Muñoz 2013), a Griffiths phase represents an extended region in the
parameter space characterized with power-laws scaling behaviour that
arises from heterogeneity at local burning patterns. In our model, some
areas may experience infrequent small fires as in the historical Amazon,
while others see more active, irregular burning influenced by localized
factors. However, we lack a rigorous result in this regard.

We observed the anticipated impact of the 2010 drought on fires in the
Amazon biome, resulting in the most extensive fire occurrences on
record, even though deforestation rates were significantly lower than in
the previous decade (Aragão et al. 2018). In contrast, other drought
years, such as the one associated with the El Niño event in 2015-2016,
led to a considerably lower number of fires compared to the 2010
drought. This discrepancy may be attributed to the nonlinear loading and
discharge cycles inherent in the dynamics of fire-forest systems.
Following fires and droughts, fuel accumulates during wetter years,
potentially yielding different effects during extreme events. Our model,
which incorporates the influence of drought using actual and predicted
temperatures to estimate the probability of ignition (\(f\)), supports
the hypothesis that nonlinear effects play a substantial role in fire
dynamics. Using the inverse of the \(p\) parameter the model also
predicts that the forest will be recovered between 13 and 19 years,
close to the 23 years suggested by Alencar (Alencar et al. 2011).

While droughts are considered the primary cause of fires in the Amazon,
deforestation is becoming the second (Aragão et al. 2018). In addition,
Cardil (2020) found that in 2018 most of the fires (85\%) were produced
in areas deforested in 2018. The fitted parameters, forced with the
ignition probability estimated from actual data are incompatible with
such a high proportion, but are close to the proportion observed in
previous years. A model incorporating explicit parameters for deforested
areas would be needed to resolve this.

The forest state in our model symbolizes flammable forest, given that
undisturbed tropical forest in the Amazon is generally considered
non-flammable with a very low probability of natural fires (Fonseca et
al. 2019). In contrast, deforested areas are flammable, and our fitted
parameters encapsulate an average of these scenarios. The proportion of
these types is changing due to increased interfaces between undisturbed
forest, human-degraded forest, and other land uses (Aragão et al. 2018).
Human-induced fires persist, originating from the transportation network
and external regions (Barlow et al. 2020). These fires can invade
standing forest, and if climate change makes forests hotter and drier,
they may sustain more extensive fires (Brando et al. 2019). Our model
incorporates these changes by varying the proportion of forest that
burns and is deforested. Thus, besides the range of variation in
\(\theta\) is higher for the scenarios with deforestation, but as we
maintain the average \(\theta\) constant at the fitted values, it is
also expected that the deforestation scenarios are far from a critical
regime as without deforestation. Additionally, as the magnitude of fires
increases, the probability of forests losing their capacity to recover
from frequent fires and droughts increases (Brando et al. 2019).

Several studies propose that a deforestation rate ranging from 20\% to
40\% in the Amazon could trigger a swift transition to non-forest
ecosystems (Nobre et al. 2016, Lovejoy and Nobre 2018). Currently,
around 20\% of the Amazon's forest has been lost since the 1960s, and
environmental signals suggest ongoing fluctuations in the system
(Lovejoy and Nobre 2018). Dynamic analysis indicates proximity to a
transition point (Saravia et al. 2018). However, our model suggests that
the fire regime will not undergo a significant change solely due to
climate change; instead, deforestation emerges as the more influential
driver. Our primary conclusion is that if deforestation and degradation
in the Amazon decrease, the region is likely to exhibit resilience
against predicted climate change, mitigating the risk of the Amazonian
tropical forest collapsing into a savanna.

\subsection{Acknowledgements}\label{acknowledgements}

LAS expresses gratitude to the Universidad Nacional de General Sarmiento
(Project 30/1139) and the Agencia Nacional de Promoción Científica y
Tecnológica (PICT 2020-SERIEA-02628) for their financial support.
Special thanks are extended to Santiago R. Doyle for providing the
computational resources utilized in this work.

\subsection{Authors' contributions}\label{authors-contributions}

LAS, BBL, and SS conceived the ideas. All authors designed methodology;
LAS collected the data; LAS wrote the code; LAS, BBL and SS analysed the
data; LAS led the writing of the manuscript. All authors contributed
critically to the drafts and gave final approval for publication.

\subsection{Data Availability
Statement}\label{data-availability-statement}

The source code and data are available at zenodo
\url{https://doi.org/10.5281/zenodo.5703638} and Github
\url{https://github.com/lsaravia/AmazonFireTippingPoints}. Remote
sensing MODIS data and climate data are available directly from NASA and
Google Earth Engine.

\subsection*{References}\label{references}
\addcontentsline{toc}{subsection}{References}

\phantomsection\label{refs}
\begin{CSLReferences}{1}{1}
\bibitem[\citeproctext]{ref-Abatzoglou2018}
Abatzoglou, J. T. et al. 2018.
\href{https://doi.org/10.1038/sdata.2017.191}{{TerraClimate}, a
high-resolution global dataset of monthly climate and climatic water
balance from 1958--2015}. - Scientific Data 5: 170191.

\bibitem[\citeproctext]{ref-Abatzoglou2019}
Abatzoglou, J. T. et al. 2019.
\href{https://doi.org/10.1029/2018GL080959}{Global {Emergence} of
{Anthropogenic Climate Change} in {Fire Weather Indices}}. - Geophysical
Research Letters 46: 326--336.

\bibitem[\citeproctext]{ref-Adams2020}
Adams, M. A. et al. 2020.
\href{https://doi.org/10.1111/gcb.15125}{Causes and consequences of
{Eastern Australia}'s 2019--20 season of mega-fires: {A} broader
perspective}. - Global Change Biology 26: 3756--3758.

\bibitem[\citeproctext]{ref-Alencar2011}
Alencar, A. et al. 2011.
\href{https://doi.org/10.1890/10-1168.1}{Temporal variability of forest
fires in eastern {Amazonia}}. - Ecological Applications 21: 2397--2412.

\bibitem[\citeproctext]{ref-Aragao2014}
Aragão, L. E. O. C. et al. 2014.
\href{https://doi.org/10.1111/brv.12088}{Environmental change and the
carbon balance of {Amazonian} forests}. - Biological Reviews 89:
913--931.

\bibitem[\citeproctext]{ref-Aragao2018}
Aragão, L. E. O. C. et al. 2018.
\href{https://doi.org/10.1038/s41467-017-02771-y}{21st {Century}
drought-related fires counteract the decline of {Amazon} deforestation
carbon emissions}. - Nature Communications 9: 536.

\bibitem[\citeproctext]{ref-Barlow2020}
Barlow, J. et al. 2020.
\href{https://doi.org/10.1111/gcb.14872}{Clarifying {Amazonia}'s burning
crisis}. - Global Change Biology 26: 319--321.

\bibitem[\citeproctext]{ref-NatureEcoEvo2020}
2020. \href{https://doi.org/10.1038/s41559-020-1119-4}{Biodiversity in
flames}. - Nature Ecology \& Evolution 4: 171--171.

\bibitem[\citeproctext]{ref-Bonachela2009}
Bonachela, J. A. and Muñoz, M. A. 2009.
\href{https://doi.org/10.1088/1742-5468/2009/09/P09009}{Self-organization
without conservation: True or just apparent scale-invariance?} - Journal
of Statistical Mechanics: Theory and Experiment 2009: P09009.

\bibitem[\citeproctext]{ref-Bowman2020}
Bowman, D. M. J. S. et al. 2020.
\href{https://doi.org/10.1038/s43017-020-0085-3}{Vegetation fires in the
{Anthropocene}}. - Nature Reviews Earth \& Environment: 1--16.

\bibitem[\citeproctext]{ref-Brando2019}
Brando, P. M. et al. 2019.
\href{https://doi.org/10.1146/annurev-earth-082517-010235}{Droughts,
{Wildfires}, and {Forest Carbon Cycling}: {A Pantropical Synthesis}}. -
Annual Review of Earth and Planetary Sciences 47: 555--581.

\bibitem[\citeproctext]{ref-Cardil2020}
Cardil, A. et al. 2020.
\href{https://doi.org/10.1088/1748-9326/abcac7}{Recent deforestation
drove the spike in {Amazonian} fires}. - Environmental Research Letters
15: 121003.

\bibitem[\citeproctext]{ref-Clark2005}
Clark, C. J. et al. 2005. Comparative seed shadows of bird-, monkey-,
and wind-dispersed trees. - Ecology 86: 2684--2694.

\bibitem[\citeproctext]{ref-Clauset2009}
Clauset, A. et al. 2009.
\href{https://doi.org/10.1137/070710111}{Power-{Law Distributions} in
{Empirical Data}}. - SIAM Review 51: 661--703.

\bibitem[\citeproctext]{ref-Csillery2010}
Csilléry, K. et al. 2010.
\href{http://dx.doi.org/10.1016/j.tree.2010.04.001}{Approximate
{Bayesian Computation} ({ABC}) in practice}. - Trends in Ecology \&
Evolution 25: 410--418.

\bibitem[\citeproctext]{ref-Csillery2012}
Csilléry, K. et al. 2012.
\href{https://doi.org/10.1111/j.2041-210X.2011.00179.x}{Abc: An {R}
package for approximate {Bayesian} computation ({ABC})}. - Methods in
Ecology and Evolution 3: 475--479.

\bibitem[\citeproctext]{ref-Dieleman2020}
Dieleman, C. M. et al. 2020.
\href{https://doi.org/10.1111/gcb.15158}{Wildfire combustion and carbon
stocks in the southern {Canadian} boreal forest: {Implications} for a
warming world}. - Global Change Biology 26: 6062--6079.

\bibitem[\citeproctext]{ref-Drossel1992}
Drossel, B. and Schwabl, F. 1992.
\href{https://doi.org/10.1103/PhysRevLett.69.1629}{Self-organized
critical forest-fire model}. - Physical Review Letters 69: 1629--1632.

\bibitem[\citeproctext]{ref-Fairman2015}
Fairman, T. A. et al. 2015. \href{https://doi.org/10.1071/WF15010}{Too
much, too soon? {A} review of the effects of increasing wildfire
frequency on tree mortality and regeneration in temperate eucalypt
forests}. - International Journal of Wildland Fire 25: 831--848.

\bibitem[\citeproctext]{ref-Fang2005}
Fang, K.-T. et al. 2005. {Design and Modeling for Computer Experiments}.

\bibitem[\citeproctext]{ref-Feng2021}
Feng, X. et al. 2021.
\href{https://doi.org/10.1038/s41586-021-03876-7}{How deregulation,
drought and increasing fire impact {Amazonian} biodiversity}. - Nature
597: 516--521.

\bibitem[\citeproctext]{ref-Fonseca2019}
Fonseca, M. G. et al. 2019.
\href{https://doi.org/10.1111/gcb.14709}{Effects of climate and land-use
change scenarios on fire probability during the 21st century in the
{Brazilian Amazon}}. - Global Change Biology 25: 2931--2946.

\bibitem[\citeproctext]{ref-Furlaud2021}
Furlaud, J. M. et al. 2021.
\href{https://doi.org/10.1111/1365-2745.13663}{Bioclimatic drivers of
fire severity across the {Australian} geographical range of giant
{Eucalyptus} forests}. - Journal of Ecology 109: 2514--2536.

\bibitem[\citeproctext]{ref-Giglio2016}
Giglio, L. et al. 2016.
\href{https://doi.org/10.1016/j.rse.2016.02.054}{The collection 6
{MODIS} active fire detection algorithm and fire products}. - Remote
Sensing of Environment 178: 31--41.

\bibitem[\citeproctext]{ref-Grassberger2002}
Grassberger, P. 2002.
\href{https://doi.org/10.1088/1367-2630/4/1/317}{Critical behaviour of
the {Drossel-Schwabl} forest fire model}. - New Journal of Physics 4:
17.

\bibitem[\citeproctext]{ref-Hansen2013}
Hansen, M. C. et al. 2013.
\href{https://doi.org/10.1126/science.1244693}{High-{Resolution Global
Maps} of 21st-{Century Forest Cover Change}}. - Science 342: 850--853.

\bibitem[\citeproctext]{ref-Hirota2011}
Hirota, M. et al. 2011.
\href{https://doi.org/10.1126/science.1210657}{Global {Resilience} of
{Tropical Forest} and {Savanna} to {Critical Transitions}}. - Science
334: 232--235.

\bibitem[\citeproctext]{ref-Hoffman2021}
Hoffman, K. M. et al. 2021.
\href{https://doi.org/10.1073/pnas.2105073118}{Conservation of {Earth}'s
biodiversity is embedded in {Indigenous} fire stewardship}. -
Proceedings of the National Academy of Sciences in press.

\bibitem[\citeproctext]{ref-Hoshen1976}
Hoshen, J. and Kopelman, R. 1976.
\href{https://doi.org/10.1103/PhysRevB.14.3438}{Percolation and cluster
distribution. {I}. {Cluster} multiple labeling technique and critical
concentration algorithm}. - Physical Review B 14: 3438--3445.

\bibitem[\citeproctext]{ref-Huang2015}
Huang, Y. et al. 2015.
\href{https://doi.org/10.1016/j.atmosenv.2015.06.002}{Sensitivity of
global wildfire occurrences to various factors in the~context of global
change}. - Atmospheric Environment 121: 86--92.

\bibitem[\citeproctext]{ref-Jensen1998}
Jensen, H. J. 1998. Self-{Organized Criticality}: {Emergent Complex
Behavior} in {Physical} and {Biological Systems}. - {Cambridge
University Press}.

\bibitem[\citeproctext]{ref-Kelly2017}
Kelly, L. T. and Brotons, L. 2017.
\href{https://doi.org/10.1126/science.aam7672}{Using fire to promote
biodiversity}. - Science 355: 1264--1265.

\bibitem[\citeproctext]{ref-Kelly2020}
Kelly, L. T. et al. 2020.
\href{https://doi.org/10.1126/science.abb0355}{Fire and biodiversity in
the {Anthropocene}}. - Science in press.

\bibitem[\citeproctext]{ref-LePage2017}
Le Page, Y. et al. 2017.
\href{https://doi.org/10.5194/esd-8-1237-2017}{Synergy between land use
and climate change increases future fire risk in {Amazon} forests}. -
Earth System Dynamics 8: 1237--1246.

\bibitem[\citeproctext]{ref-Lenton2013}
Lenton, T. M. and Williams, H. T. P. 2013.
\href{https://doi.org/10.1016/j.tree.2013.06.001}{On the origin of
planetary-scale tipping points}. - Trends in Ecology \& Evolution 28:
380--382.

\bibitem[\citeproctext]{ref-Levin1996}
Levin, S. A. and Durret, R. 1996.
\href{https://doi.org/10.1098/rstb.1996.0145}{From individuals to
epidemics}. - Philosophical Transactions of the Royal Society of London.
Series B 351: 1615--1621.

\bibitem[\citeproctext]{ref-Lovejoy2018}
Lovejoy, T. E. and Nobre, C. 2018.
\href{https://doi.org/10.1126/sciadv.aat2340}{Amazon {Tipping Point}}. -
Science Advances 4: eaat2340.

\bibitem[\citeproctext]{ref-Marco2011}
Marco, D. E. et al. 2011.
\href{https://doi.org/10.1111/j.1600-0587.2010.06477.x}{Comparing short
and long-distance dispersal: Modelling and field case studies}. -
Ecography 34: 671--682.

\bibitem[\citeproctext]{ref-Masson-Delmotte2021}
2021. Summary for policymakers. - In: Masson-Delmotte, V. et al. (eds),
Climate {Change} 2021: {The Physical Science Basis}. {Contribution} of
{Working Group I} to the {Sixth Assessment Report} of the
{Intergovernmental Panel} on {Climate Change}. {Cambridge University
Press}, in press.

\bibitem[\citeproctext]{ref-Meinshausen2011}
Meinshausen, M. et al. 2011.
\href{https://doi.org/10.1007/s10584-011-0156-z}{The {RCP} greenhouse
gas concentrations and their extensions from 1765 to 2300}. - Climatic
Change 109: 213--241.

\bibitem[\citeproctext]{ref-Moretti2013}
Moretti, P. and Muñoz, M. A. 2013.
\href{https://doi.org/10.1038/ncomms3521}{Griffiths phases and the
stretching of criticality in brain networks}. - Nature Communications 4:
2521.

\bibitem[\citeproctext]{ref-Moss2010}
Moss, R. H. et al. 2010. \href{https://doi.org/10.1038/nature08823}{The
next generation of scenarios for climate change research and
assessment}. - Nature 463: 747--756.

\bibitem[\citeproctext]{ref-Nathan2008}
Nathan, R. et al. 2008.
\href{https://doi.org/10.1016/j.tree.2008.08.003}{Mechanisms of
long-distance seed dispersal}. - Trends in Ecology \& Evolution 23:
638--647.

\bibitem[\citeproctext]{ref-Newman2005}
Newman, M. E. J. 2005.
\href{https://doi.org/10.1080/00107510500052444}{Power laws, {Pareto}
distributions and {Zipf}'s law}. - Contemporary Physics 46: 323--351.

\bibitem[\citeproctext]{ref-Nicoletti2023}
Nicoletti, G. et al. 2023.
\href{https://doi.org/10.1016/j.isci.2023.106181}{The emergence of
scale-free fires in {Australia}}. - iScience in press.

\bibitem[\citeproctext]{ref-Nobre2016}
Nobre, C. A. et al. 2016.
\href{https://doi.org/10.1073/pnas.1605516113}{Land-use and climate
change risks in the {Amazon} and the need of a novel sustainable
development paradigm}. - Proceedings of the National Academy of Sciences
113: 10759--10768.

\bibitem[\citeproctext]{ref-Nolan2020}
Nolan, R. H. et al. 2020.
\href{https://doi.org/10.1111/gcb.14987}{Causes and consequences of
eastern {Australia}'s 2019--20 season of mega-fires}. - Global Change
Biology 26: 1039--1041.

\bibitem[\citeproctext]{ref-Oborny2007}
Oborny, B. et al. 2007. Survival of species in patchy landscapes:
Percolation in space and time. - In: Scaling {Biodiversity}. {Cambridge
University Press}, pp. 409--440.

\bibitem[\citeproctext]{ref-Palmieri2020}
Palmieri, L. and Jensen, H. J. 2020.
\href{https://doi.org/10.3389/fphy.2020.00257}{The {Forest Fire Model}:
{The Subtleties} of {Criticality} and {Scale Invariance}}. - Frontiers
in Physics 8: 257.

\bibitem[\citeproctext]{ref-Pausas2021}
Pausas, J. G. and Keeley, J. E. 2021.
\href{https://doi.org/10.1002/fee.2359}{Wildfires and global change}. -
Frontiers in Ecology and the Environment 19: 387--395.

\bibitem[\citeproctext]{ref-Pedersen2019}
Pedersen, E. J. et al. 2019.
\href{https://doi.org/10.7717/peerj.6876}{Hierarchical generalized
additive models in ecology: An introduction with mgcv}. - PeerJ 7:
e6876.

\bibitem[\citeproctext]{ref-Pivello2021}
Pivello, V. R. et al. 2021.
\href{https://doi.org/10.1016/j.pecon.2021.06.005}{Understanding
{Brazil}'s catastrophic fires: {Causes}, consequences and policy needed
to prevent future tragedies}. - Perspectives in Ecology and Conservation
in press.

\bibitem[\citeproctext]{ref-Pueyo2007a}
Pueyo, S. 2007.
\href{https://doi.org/10.1007/s10584-006-9134-2}{Self-{Organised
Criticality} and the {Response} of {Wildland Fires} to {Climate
Change}}. - Climatic Change 82: 131--161.

\bibitem[\citeproctext]{ref-Pueyo2010}
Pueyo, S. et al. 2010.
\href{https://doi.org/10.1111/j.1461-0248.2010.01497.x}{Testing for
criticality in ecosystem dynamics: The case of {Amazonian} rainforest
and savanna fire}. - Ecology Letters 13: 793--802.

\bibitem[\citeproctext]{ref-Ratz1995}
Ratz, A. 1995. \href{https://doi.org/10.1071/wf9950025}{Long-{Term
Spatial Patterns Created} by {Fire}: A {Model Oriented Towards Boreal
Forests}}. - International Journal of Wildland Fire 5: 25--34.

\bibitem[\citeproctext]{ref-Sanderson2020}
Sanderson, B. M. and Fisher, R. A. 2020.
\href{https://doi.org/10.1038/s41558-020-0707-2}{A fiery wake-up call
for climate science}. - Nature Climate Change 10: 175--177.

\bibitem[\citeproctext]{ref-Saravia2018a}
Saravia, L. A. et al. 2018.
\href{https://doi.org/10.1038/s41598-018-36120-w}{Power laws and
critical fragmentation in global forests}. - Scientific Reports 8:
17766.

\bibitem[\citeproctext]{ref-Schertzer2014}
Schertzer, E. et al. 2014.
\href{https://doi.org/10.1007/s00285-014-0757-z}{Implications of the
spatial dynamics of fire spread for the bistability of savanna and
forest}. - Journal of Mathematical Biology 70: 329--341.

\bibitem[\citeproctext]{ref-Seri2012}
Seri, E. et al. 2012. \href{https://doi.org/10.1086/668125}{Neutral
{Dynamics} and {Cluster Statistics} in a {Tropical Forest}.} - The
American Naturalist 180: E161--E173.

\bibitem[\citeproctext]{ref-SilvaJunior2020}
Silva Junior, C. H. L. et al. 2020.
\href{https://doi.org/10.1038/s41559-020-01368-x}{The {Brazilian Amazon}
deforestation rate in 2020 is the greatest of the decade}. - Nature
Ecology \& Evolution: 1--2.

\bibitem[\citeproctext]{ref-Sole2006}
Solé, R. V. and Bascompte, J. 2006. Self-organization in {Complex
Ecosystems}. - {Princeton University Press}.

\bibitem[\citeproctext]{ref-Stauffer1994}
Stauffer, D. and Aharony, A. 1994. Introduction {To Percolation Theory}.
- {Tayor \& Francis}.

\bibitem[\citeproctext]{ref-Staver2011}
Staver, A. C. et al. 2011.
\href{https://doi.org/10.1126/science.1210465}{The {Global Extent} and
{Determinants} of {Savanna} and {Forest} as {Alternative Biome States}}.
- Science 334: 230--232.

\bibitem[\citeproctext]{ref-Steel2021}
Steel, Z. L. et al. 2021.
\href{https://doi.org/10.1098/rspb.2020.3202}{Quantifying pyrodiversity
and its drivers}. - Proceedings of the Royal Society B: Biological
Sciences 288: 20203202.

\bibitem[\citeproctext]{ref-Taylor2012}
Taylor, K. E. et al. 2012.
\href{https://doi.org/10.1175/BAMS-D-11-00094.1}{An {Overview} of
{CMIP5} and the {Experiment Design}}. - Bulletin of the American
Meteorological Society 93: 485--498.

\bibitem[\citeproctext]{ref-Thonicke2010}
Thonicke, K. et al. 2010.
\href{https://doi.org/10.5194/bg-7-1991-2010}{The influence of
vegetation, fire spread and fire behaviour on biomass burning and trace
gas emissions: Results from a process-based model}. - Biogeosciences 7:
1991--2011.

\bibitem[\citeproctext]{ref-Thrasher2012}
Thrasher, B. et al. 2012.
\href{https://doi.org/10.5194/hess-16-3309-2012}{Technical {Note}:
{Bias} correcting climate model simulated daily temperature extremes
with quantile mapping}. - Hydrology and Earth System Sciences 16:
3309--3314.

\bibitem[\citeproctext]{ref-Turco2018}
Turco, M. et al. 2018.
\href{https://doi.org/10.1038/s41467-018-05250-0}{Skilful forecasting of
global fire activity using seasonal climate predictions}. - Nature
Communications 9: 2718.

\bibitem[\citeproctext]{ref-Uhl1990}
Uhl, C. and Kauffman, J. B. 1990.
\href{https://doi.org/10.2307/1940299}{Deforestation, {Fire
Susceptibility}, and {Potential Tree Responses} to {Fire} in the
{Eastern Amazon}}. - Ecology 71: 437--449.

\bibitem[\citeproctext]{ref-Vuong1989}
Vuong, Q. H. 1989. \href{https://doi.org/10.2307/1912557}{Likelihood
{Ratio Tests} for {Model Selection} and {Non-Nested Hypotheses}}. -
Econometrica 57: 307--333.

\bibitem[\citeproctext]{ref-Wei2020}
Wei, F. et al. 2020. \href{https://doi.org/10.1111/gcb.15190}{Nonlinear
dynamics of fires in {Africa} over recent decades controlled by
precipitation}. - Global Change Biology 26: 4495--4505.

\bibitem[\citeproctext]{ref-Wood2017}
Wood, S. N. 2017. Generalized {Additive Models}: {An Introduction} with
{R}, {Second Edition}. - {CRC Press}.

\bibitem[\citeproctext]{ref-Zinck2009}
Zinck, R. D. and Grimm, V. 2009.
\href{https://doi.org/10.1086/605959}{Unifying wildfire models from
ecology and statistical physics.} - The American naturalist 174:
E170--85.

\bibitem[\citeproctext]{ref-Zinck2011}
Zinck, R. D. et al. 2011.
\href{https://doi.org/10.1086/662675}{Understanding {Shifts} in
{Wildfire Regimes} as {Emergent Threshold Phenomena}.} - The American
Naturalist 178: E149--E161.

\end{CSLReferences}

\end{document}
